\documentclass[twocolumn]{jsbook}
\bibliographystyle{jplain}
\usepackage[jis2004]{otf}
\title{ルビリオ当て字典}
\author{
さとしん
}

\usepackage{ulem}
\usepackage{txfonts}
\usepackage{enumerate}
\usepackage{enumitem}
\usepackage{url}
%\usepackage{autobreak}

\newcommand{\ccite}[1]{《#1》}

\newcommand{\exsentence}[3]{\item 《#1》\ #2\ (#3).}
\newcommand{\exlist}[1]{
\begin{enumerate}[label= \raise0.2ex\hbox{\textcircled{\scriptsize{\arabic*}}}]
        #1
\end{enumerate}
}

\newcommand{\NoLabSectionTable}[1]{
    \section*{#1}
    \addcontentsline{toc}{section}{#1}
}

\begin{document}
\maketitle
\tableofcontents

\chapter{ルビリオ当て字典}

\noindent
\begin{enumerate}
% ======================================
\NoLabSectionTable{あ}
% ======================================

    \item \textbf{アダムヘッド}【聖断罪】『聖断罪ドロシー』(十文字青)
    \item \textbf{あなた}【汝】
    \exlist{
    	\exsentence{修道少女と偶像少女}{大量爆撃のお祈り「\uline{汝}に幸せありますように☆」}{cosMo(暴走P)、2012}
    }
    \item \textbf{あぽろ11号}【人間の月侵攻】
    \exlist{
    	\exsentence{狂気に満ちた不可視の珠 feat.~ytr}{全てを変えたのは\uline{人間の月侵攻}}{ytr、2010}
    }
    \item \textbf{あめ}【涙】
    \exlist{
    	\exsentence{修道少女と偶像少女}{鳴り止まない念と 降り続く後悔の\uline{涙}だけ}{cosMo(暴走P)、2012}
    }
    \item \textbf{あらし}【暴風雨】
    \exlist{
    	\exsentence{首が落ちた話}{丈の高い高粱が、まるで\uline{暴風雨}にでも遇ったようにゆすぶれたり、そのゆすぶれている穂の先に、銅のような太陽が懸っていたりした事は、不思議なくらいはっきり覚えている。}{芥川龍之介、1918}
    }
    \item \textbf{ありがとうございます}【膣内射精感謝】
    \exlist{
    	\exsentence{魔法少女催眠パコパコーズ2}{\uline{膣内射精感謝}~$\varheartsuit$}{santa、2017}
    }
    \item \textbf{アルティメットキャノン}【三重大祓砲】『呪術廻戦』(芥見下々)
    \item \textbf{?}【あれ?】
    \exlist{
    	\exsentence{修道少女と偶像少女}{\uline{?}}{cosMo(暴走P)、2012}
    }
    \item \textbf{アンブラハンズ}【祈手】
    \exlist{
    	\exsentence{メイドインアビス}{既に『\uline{祈手}』を三人も消費してしまった}{つくしあきひと、2012-}
    	\exsentence{\url{https://x.com/gerogero00001/status/1321087847286349824}}{『\uline{祈手}』は異常者の集まりだからだ}{\texttt{@gerogero00001}、2020}
    }
    
% ======================================
\NoLabSectionTable{い}
% ======================================

    \item \textbf{イェーガー}【狩人】
    \exlist{
        \exsentence{紅蓮の弓矢}{囚われた屈辱は反撃の嚆矢だ城壁の其の彼方獲物を屠る≪\uline{狩人}≫}{Revo、2014}
    }
    \item \textbf{いのり}【感謝】
    \exlist{
        \exsentence{修道少女と偶像少女}{ああ忘れてたよ ゴメン→ 形式だけの\uline{感謝}≪物欲薄イシ≫ ←平和な証でしょ}{cosMo(暴走P)、2012}
    }
    \item \textbf{いま}【不本意な現状】
    \exlist{
        \exsentence{紅蓮の弓矢}{≪\uline{不本意な現状}≫を変えるのは戦う覚悟だ}{Revo、2014}
    }
    \item \textbf{いま}【現代】
    \exlist{
        \exsentence{リアル初音ミクの消失}{\uline{現代}に生まれた悦びを}{cosMo(暴走P)}{2018}
    }
    \item \textbf{いもっこ}【妹】
    \exlist{
        \exsentence{秋田妹!えびなちゃん}{秋田\uline{妹}!えびなちゃん}{サンカクヘッド、2016-}
    }
    \item \textbf{イドフロント}【前線基地】
    \exlist{
        \exsentence{メイドインアビス}{深界五層下部 \uline{前線基地}}{つくしあきひと、2012-}
    }
    \item \textbf{イリュージョンシーカー}【狂視調律】
    \exlist{
        \exsentence{東方永夜抄 〜 Imperishable Night.}{狂視「\uline{狂視調律}(イリュージョンシーカー)」}{ZUN、2004}
    }
    \item \textbf{イル}【治癒】
    \exlist{
        \exsentence{小説の小説}{\uline{治癒}系のインチキは初めてですか?}{似鳥鶏、2022}
    }
%    \item \textbf{いろいろ}【種々】「国勢調査ビラ」(東京市役所、1920?)
    \item \textbf{インシネレーター}【火葬砲】
    \exlist{
        \exsentence{メイドインアビス}{\uline{火葬砲}(インシネレーター)}{つくしあきひと、2012-}
    }

% ======================================
\NoLabSectionTable{う}
% ======================================

    \item \textbf{ヴァナディース}【戦姫】
    \exlist{
        \exsentence{魔弾の王と戦姫}{魔弾の王と\uline{戦姫}}{川口士、2011-}
    }
    \item \textbf{ヴィオラ}【五重奏】
    \exlist{
        \exsentence{呪術廻戦}{\uline{五重奏}}{芥見下々、2018-}
    }
    \item \textbf{ウホッ}【火\UTF{7130}】
    \exlist{
        \exsentence{小説の小説}{\uline{火\UTF{7130}}!}{似鳥鶏、2022}
    }
    \item \textbf{ウホホッ}【拡散火\UTF{7130}】
    \exlist{
        \exsentence{小説の小説}{ウホッ!\uline{拡散火\UTF{7130}}!拡散火\UTF{7130}・強化!と吠えながら杖を振りかざすゴリラはやはり強い。}{似鳥鶏、2022}
    }
    \item \textbf{ウホホホーイ}【拡散火\UTF{7130}・強化】
    \exlist{
        \exsentence{小説の小説}{ウホッ!拡散火\UTF{7130}!\uline{拡散火\UTF{7130}・強化}!と吠えながら杖を振りかざすゴリラはやはり強い。}{似鳥鶏、2022}
    }
    \item \textbf{ウルトラキャノン}【大祓砲】
    \exlist{
        \exsentence{呪術廻戦}{\uline{大祓砲}}{芥見下々、2018-}
    }
    \item \textbf{ウルトラシールド}【剣山盾】
    \exlist{
        \exsentence{呪術廻戦}{\uline{剣山盾}}{芥見下々、2018-}
    }
    \item \textbf{ウルトラスピン}【絶技抉剔】
    \exlist{
        \exsentence{呪術廻戦}{\uline{絶技抉剔}}{芥見下々、2018-}
    }

% ======================================
\NoLabSectionTable{え}
% ======================================

    \item \textbf{エクソシスト}【祓魔師】
    \exlist{
        \exsentence{青の祓魔師}{青の\uline{祓魔師}}{加藤和恵、2009-}
    }
    \item \textbf{エンジェル}【尻拭き輪】
    \exlist{
        \exsentence{小説の小説}{むこうの世界の人に失礼な気がした上、あちらにいる時は日常化しすぎていたせいもあって記録には書かなかったのだが、むこうの世界では洗浄機能付き便座はもとより水洗トイレもトイレットペーパーもなく、トイレは「便穴」、尻を拭くのは荒縄でできた「\uline{尻拭き輪}」、もちろん共用で使い回しだった。}{似鳥鶏、2022}
    }

% ======================================
\NoLabSectionTable{お}
% ======================================

    \item \textbf{オーバード}【奈落の至宝】
    \exlist{
        \exsentence{メイドインアビス}{\uline{奈落の至宝}}{つくしあきひと、2012-}
    }
    \item \textbf{オキシドール}【果実酒】
    \exlist{
        \exsentence{小説の小説}{川鰈の丸焼きや葉桃の\uline{果実酒}漬けなど、御馳走も振る舞ってもらった。}{似鳥鶏、2022}
    }
    \item \textbf{おしり【相棒】}
    \exlist{
        \exsentence{映画おしりたんてい さらば愛しき相棒よ}{映画おしりたんてい さらば愛しき\uline{相棒}よ}{トロル、2024}
    }
    \item \textbf{オッパイ}【爆発】
    \exlist{
        \exsentence{小説の小説}{\uline{爆発}!}{似鳥鶏、2022}
    }
    \item \textbf{オッパイパイ}【爆発・連撃】
    \exlist{
        \exsentence{小説の小説}{\uline{爆発・連撃}!}{似鳥鶏、2022}
    }
    \item \textbf{オッパイホシイヨーウ}【爆発・多重強化】
    \exlist{
        \exsentence{小説の小説}{\uline{爆発・多重強化}!}{似鳥鶏、2022}
    }
    \item \textbf{おと}【音色】
    \exlist{
        \exsentence{Relay〜杜の詩}{ピアノの\uline{音色}が今も胸に響く}{桑田佳祐、2023}
    }
%    \item \textbf{おとこ}【強盗犯】『雪女と蟹を食う』(Gino0808)
    \item \textbf{おとこ}【神父】
    \exlist{
        \exsentence{修道少女と偶像少女}{掴まれた肩 背後には若い\uline{神父}}{cosMo(暴走P)、2012}
    }
    \item \textbf{おわり}【際】
    \exlist{
        \exsentence{リアル初音ミクの消失}{せめて\uline{際}に見る世界が暖かいものでありますように}{cosMo(暴走P)、2018}
    }
%    \item \textbf{おんな}【セレブ妻】『雪女と蟹を食う』(Gino0808)
    \item \textbf{おんな}【私】
    \exlist{
        \exsentence{\url{https://x.com/agu_knzm/status/1515535806689021953}}{あなたの推しは、隣の席の\uline{私}なのである}{\texttt{@agu\_knzm}}
    }

% ======================================
\NoLabSectionTable{か}
% ======================================

    \item \textbf{ガーゼ}【葉桃】
    \exlist{
        \exsentence{小説の小説}{川鰈の丸焼きや\uline{葉桃}の果実酒漬けなど、御馳走も振る舞ってもらった。}{似鳥鶏、2022}
    }
    \item \textbf{ガソリン}【燃料】
    \exlist{
        \exsentence{ギラギラ}{孤独は\uline{燃料} 卑屈な町を行く}{てにをは、2021}
    }
    \item \textbf{かたち}【形式】
    \exlist{
        \exsentence{修道少女と偶像少女}{ああ忘れてたよ ゴメン→ \uline{形式}だけの感謝≪物欲薄イシ≫ ←平和な証でしょ}{cosMo(暴走P)、2012}
    }
    \item \textbf{かたち}【記録】『神々の祈り』(憐歌、2009)
    \item \textbf{からだ}【体躯】『Sadistic.Music∞Factory』(cosMo(暴走P)、2013)

% ======================================
\NoLabSectionTable{き}
% ======================================

    \item \textbf{きゃら}【設定】『初音ミクとあそぼぅ!!』(cosMo(暴走P)、2010)
    \item \textbf{きょろきょろ}【左視右顧】(\UTF{56FE}\UTF{5E08}、2023)
    \item \textbf{キラーフィッシュ}【牙魚】『小説の小説』(似鳥鶏、2022)

% ======================================
\NoLabSectionTable{く}
% ======================================

    \item \textbf{くうき}【善悪】『R.I.Pゴシップの海』(cosMo(暴走P)、2020)
    \item \textbf{くろ}【黒鋼】『黒鋼の魔紋修復士』(嬉野秋彦)
    \item \textbf{クロスロード}【十字路】『迷宮の十字路』(2003)
    \item \textbf{ケルメ}【間】『小説の小説』(似鳥鶏、2022)

% ======================================
\NoLabSectionTable{け}
% ======================================

    \item \textbf{けんかわかれ}【疎蔭】(\UTF{56FE}\UTF{5E08}、2023)
    \item \textbf{マジシャン}【奇術師】『銀翼の奇術師』(2004)
    \item \textbf{マンネリ}【倦怠】『シャノワールの冒険書』

% ======================================
\NoLabSectionTable{こ}
% ======================================

    \item \textbf{こえ}【言葉】『0→∞への跳動』
    \item \textbf{こえ}【念】『修道少女と偶像少女』(cosMo(暴走P)、2012)
    \item \textbf{ここ}【この杜】『Relay〜杜の詩』
    \item \textbf{ここ}【中学】https://twitter.com/kiriillust/status/1513885067185172495
    \item \textbf{こころ}【像】『Sadistic.Music∞Factory』(cosMo(暴走P)、2013)
    \item \textbf{こちら}【東側】『小説の小説』(似鳥鶏、2022) 
    \item \textbf{こと}【補習】
    \item \textbf{こま}【高麗】◇
    \item \textbf{コルテメルケル}【医療施設】『小説の小説』(似鳥鶏、2022)
    \item \textbf{コルテラ}【診療所】『小説の小説』(似鳥鶏、2022)
    \item \textbf{これが真相である}【ルビでそう述べても】『小説の小説』(似鳥鶏、2022)

% ======================================
\NoLabSectionTable{さ}
% ======================================

    \item \textbf{ザーメンパレット}【小学生褐色肌】『魔法少女催眠パコパコーズ2』(santa、2017)
    \item \textbf{さおとめ}【五月女】苗字
    \item \textbf{サブマリン}【魚影】『黒鉄の魚影』(2023)
    \item \textbf{さるすべり}【百日紅】◇

% ======================================
\NoLabSectionTable{し}
% ======================================

    \item \textbf{シーカーキャンプ}【監視基地】『メイドインアビス』(つくしあきひと)
    \item \textbf{システム}【機構】『0→∞への跳動』
    \item \textbf{しぶき}【飛沫】◇
    \item \textbf{じゃ}【龍】「龍踊」
    \item \textbf{しゅうのなかからひとりえらびだす}【採擢】(\UTF{56FE}\UTF{5E08}、2023)
    \item \textbf{しょうどう}【殺意】
    \begin{enumerate}[label= \raise0.2ex\hbox{\textcircled{\scriptsize{\arabic*}}}]
        \item \ccite{紅蓮の弓矢}{迸る≪\uline{殺意}≫に其の身を灼きながら黄昏に緋を穿つーー(Revo、2014)}
        \item \ccite{紅蓮の弓矢}{止めどなき≪\uline{殺意}≫に其の身を侵されながら宵闇に紫を運ぶ(Revo、2014)}
    \end{enumerate}
    \item \textbf{ジョリー・ロジャー}【棺】『紺碧の棺』(2007)

% ======================================
\NoLabSectionTable{す}
% ======================================

    \item \textbf{スイッチオン}【PC起動】\ccite{修道少女と偶像少女}<<\uline{PC起動}>>(cosMo(暴走P)、2012)
    \item \textbf{スタンプ}【膣内射精】『魔法少女催眠パコパコーズ2』(2017、santa)
    \item \textbf{ストラテジー}【陰謀】『水平線上の陰謀』
    \item \textbf{ストリート}【街】『ベイカー街の亡霊』(2002)
    \item \textbf{スナイパー}【狙撃手】『異次元の狙撃手』(2014)

% ======================================
\NoLabSectionTable{せ}
% ======================================

    \item \textbf{せかい}【宇宙】『初音天地開闢神話』
    \item \textbf{ゼロ・ランブル}【無魔術乱舞】『くすぐり闘士の無魔術乱舞』
    \item \textbf{せんそう}【宗教戦争】『修道少女と偶像少女』(cosMo(暴走P)、2012)

% ======================================
\NoLabSectionTable{そ}
% ======================================

    \item \textbf{ソードオプション}【刀源解放】『呪術廻戦』(芥見下々)
    \item \textbf{そっかわ}【\UTF{7C0F}津川】◇地名
    \item \textbf{そら}【秋空】『隊友』
    \item \textbf{それ}【スクール水着】\ccite{らき☆すた1}{こなたサン\uline{スクール水着}は本気っスか!?(美水かがみ、2005)}

% ======================================
\NoLabSectionTable{た}
% ======================================

    \item \textbf{ターゲット}【標的】『14番目の標的』(1998)
    \item \textbf{だいだい}【世々】『武蔵野話』(斎藤鶴磯、1815)
    \item \textbf{たうげ}【嶺】『武蔵野話』(斎藤鶴磯、1815)
    \item \textbf{たくさん}【大量爆撃】\ccite{修道少女と偶像少女}大量爆撃のお祈り「\uline{汝(なんじ)}に幸せありますように☆」(cosMo(暴走P)、2012)
    \item \textbf{タシカナル}【正確】「国勢調査ポスター」(1920)
    \item \textbf{たちひろし}【斬撃広域】\url{https://twitter.com/tanimikitakane/status/1033557276613783553}
    \item \textbf{たばこ}【煙草】◇
    \item \textbf{たべもの}【音楽】『Sadistic.Music∞Factory』(cosMo(暴走P)、2013)
\section*{ち}
    \item \textbf{チェイサー}【追跡者】『漆黒の追跡者』(2009)
    \item \textbf{チェリッシュ}【秘有】『踊る地平線:08 しっぷ・あほうい!』(谷譲次、1934)
    \item \textbf{ちかごろ}【近会】『武蔵野話』(斎藤鶴磯、1815)
    \item \textbf{ちかた}【千門】◇地名
    \item \textbf{ちから}【能力】『神々の祈り』(憐歌、2009)
    \item \textbf{チカラヲアハス}【協力】「国勢調査ポスター」(1920)
    \item \textbf{ちぎやう}【采地】『武蔵野話』(斎藤鶴磯、1815)
    \item \textbf{チンク}【支那人】『踊る地平線:08 しっぷ・あほうい!』(谷譲次、1934)
    \item \textbf{チンポケース}【膣内形状】\ccite{魔法少女催眠パコパコーズ2}{君もすぐにイリヤと同じ俺専用\uline{膣内形状}にしてあげるからね$\varheartsuit$}(santa、2017)
\section*{つ}
\section*{て}
    \item \textbf{ディスカーダー}【喪心創痍】
    \item \textbf{ディモチヴィエイション}【喪心喪意】
    \item \textbf{デブネコ}【他動骸骨】『小説の小説』(似鳥鶏、2022)
    \item \textbf{てら}【院】『武蔵野話』(斎藤鶴磯、1815)
    \item \textbf{テレキネシス}【念動力】『とある魔術の禁書目録』(鎌池和馬)
    \item \textbf{テレメスメリズム}【月兎遠隔催眠術】『東方永夜抄』(ZUN)
    \item \textbf{デンジャラス}【危険】『初音ミクとあそぼぅ!!』(cosMo(暴走P)、2010)
    \item \textbf{てんだい}【台教】『武蔵野話』(斎藤鶴磯、1815)
    \item \textbf{テンタサン}【誘惑】『黒死館殺人事件』(小栗虫太郎、1934)
\section*{と}
    \item \textbf{どうぐ}【凶器】\ccite{紅蓮の弓矢}{≪\uline{凶器}≫でも技術でもない(Revo、2014)}
    \item \textbf{トウスト}【焼麺麭】『踊る地平線:08 しっぷ・あほうい!』(谷譲次、1934)
    \item \textbf{どうらく}【逸放】(\UTF{56FE}\UTF{5E08}、2023)
    \item \textbf{とき}【刻】『初音天地開闢神話』
    \item \textbf{ところてん}【心太】◇
    \item \textbf{ところのもの}【土人】『武蔵野話』(斎藤鶴磯、1815)
    \item \textbf{としつき}【星霜】『武蔵野話』(斎藤鶴磯、1815)
    \item \textbf{ともし}【燭燈】『東京八景』(太宰治)
    \item \textbf{トラウリッヒ}【悲愁】『黒死館殺人事件』(小栗虫太郎、1934)
    \item \textbf{ドラッグ}【脳汁】『R.I.Pゴシップの海』(cosMo(暴走P)、2020)
    \item \textbf{トラブル}【不運】『シャノワールの冒険書』
    \item \textbf{ドラミングビート}【激震掌】『呪術廻戦』(芥見下々)
    \item \textbf{とりゐ}【華表】『武蔵野話』(斎藤鶴磯、1815)
    \item \textbf{ナイトメア}【悪夢】『純黒の悪夢』(2016)
\section*{な}
    \item \textbf{なかたがひ}【矛盾】『武蔵野話』(斎藤鶴磯、1815)
    \item \textbf{なかだし}【腟内射精】『魔法少女催眠パコパコーズ2』(santa、2017)
    \item \textbf{なぬし}【庄家】『武蔵野話』(斎藤鶴磯、1815)
\section*{に}
    \item \textbf{ニイルカグ}【逃げるな!】『小説の小説』(似鳥鶏、2022)
\section*{ぬ}
\section*{ね}
    \item \textbf{ネザースター}【奈落の星】『メイドインアビス』(つくしあきひと)
    \item \textbf{ネクロマンサー}【死霊操術師】『小説の小説』(似鳥鶏、2022)
\section*{の}
    \item \textbf{濃厚ザーメンソース}【優勝賞品】\ccite{魔法少女催眠パコパコーズ2}{おらッ受け取れ!!!\uline{優勝賞品}だぞォ$\varheartsuit$}(santa、2017)
    \item \textbf{濃厚ザーメンソース}【優勝商品】『魔法少女催眠パコパコーズ2』(santa、2017)
\section*{は}
    \item \textbf{ハートブレイク}【傷心】『ジュリアに傷心』
    \item \textbf{ハイドギヴァー}【地臥せり】『メイドインアビス』(つくしあきひと)
    \item \textbf{パイロ・アルト}【山の手】『踊る地平線:08 しっぷ・あほうい!』(谷譲次、1934)
    \item \textbf{ハウス}【家】『踊る地平線:08 しっぷ・あほうい!』(谷譲次、1934)
    \item \textbf{はかりしれないいいみ}【\UTF{8CFE}旨】(\UTF{56FE}\UTF{5E08}、2023)
    \item \textbf{はざま}【中立地帯】『R.I.Pゴシップの海』(cosMo(暴走P)、2020)
    \item \textbf{場所}【スタジオ】『Relay〜杜の詩』
    \item \textbf{はた}【旌旗】『武蔵野話』(斎藤鶴磯、1815)
    \item \textbf{はたけ}【葱畑】『初音ミクとあそぼぅ!!』(cosMo(暴走P)、2010)
    \item \textbf{パテル}【父】『黒死館殺人事件』(小栗虫太郎、1934)
    \item \textbf{はやりうた}【謡語】(\UTF{56FE}\UTF{5E08}、2023)
    \item \textbf{はやりかぜ}【流行性感冒】『鈴木家文書』(松井村役場、1918)
    \item \textbf{バラ}【絳蕊】(\UTF{56FE}\UTF{5E08}、2023)
    \item \textbf{パレイド}【行列】『踊る地平線:08 しっぷ・あほうい!』(谷譲次、1934)
    \item \textbf{ハローアビス}【奈落の連環】『メイドインアビス』(つくしあきひと)
    \item \textbf{ばをかきみだしこんらんさせる}【掠亂】(\UTF{56FE}\UTF{5E08}、2022)
\section*{ひ}
    \item \textbf{ヒエラ・グラフィコス}【魔紋修復士】『黒鋼の魔紋修復士』(嬉野秋彦)
    \item \textbf{ピコ}【寸】『小説の小説』(似鳥鶏、2022)
    \item \textbf{ビジョン}【追尾】『呪術廻戦』(芥見下々)
    \item \textbf{ひと}【人間】『Relay〜杜の詩』
    \item \textbf{ひのたま}【炎光】『武蔵野話』(斎藤鶴磯、1815)
    \item \textbf{ひまわり}【向日葵】◇
    \item \textbf{ひみつ}【裏口】https://www.nicovideo.jp/watch/sm22839044(2014、伯方)
\section*{ふ}
    \item \textbf{フィルタ}【分別】『修道少女と偶像少女』(cosMo(暴走P)、2012)
    \item \textbf{ひやくしやう}【農民】『武蔵野話』(斎藤鶴磯、1815)
    \item \textbf{フィスト}【拳】『紺青の拳』(2019)
    \item \textbf{フーガ}【開】『呪術廻戦』(芥見下々)
    \item \textbf{ブーストオン}【推力加算】『呪術廻戦』(芥見下々)
    \item \textbf{ブギウギ}【不義遊戯】『呪術廻戦』(芥見下々)
    \item \textbf{ふぐ}【河豚】◇
    \item \textbf{ふつかよい}【宿酔】『檸檬』(梶井基次郎)
    \item \textbf{プライベート・アイ}【探偵】『絶海の探偵』(2013)
    \item \textbf{プラトニックラブ}【種付けするだけ】\ccite{魔法少女催眠パコパコーズ2}{\uline{種付けするだけ}の関係に留まらず}(santa、2017)
    \item \textbf{プリケツ}【魔族】『小説の小説』(似鳥鶏、2022)
    \item \textbf{プリプリ}【魔法】『小説の小説』(似鳥鶏、2022)
    \item \textbf{ふるきかきもの}【古証文】『武蔵野話』(斎藤鶴磯、1815)
    \item \textbf{フルーツ・スキャンダル}【果実大恋愛】『君たちキウイ・パパイヤ・マンゴだね』
    \item \textbf{フルスコア}【楽譜】『戦慄の楽譜』(2008)
    \item \textbf{ブレス}【息】『小説の小説』(似鳥鶏、2022)
    \item \textbf{プロジェクター}【映写機】『うつす』(中井正一、1932)
\section*{へ}
    \item \textbf{ヘイルヘックス}【呪詛船団】『メイドインアビス』(2012-、つくしあきひと)
    \item \textbf{ヘヴン}【便穴】『小説の小説』(2022、似鳥鶏)
    \item \textbf{へや}【室】『四方山の話 : 随筆』(三浦藤作、1927)
\section*{ほ}
    \item \textbf{ボウテ}【小舟】『踊る地平線:08 しっぷ・あほうい!』(谷譲次、1934)
    \item \textbf{ボウテ}【短舟】『踊る地平線:08 しっぷ・あほうい!』(谷譲次、1934)
    \item \textbf{ポコペン}【地球】
    \item \textbf{ポロット}【大通り】『踊る地平線:08 しっぷ・あほうい!』(谷譲次、1934)
\section*{ま}
    \item \textbf{マアル・テアネブロウゾ}【暗黒の海】『踊る地平線:08 しっぷ・あほうい!』(谷譲次、1934)
    \item \textbf{マインドシェイカー}【赤眼催眠】『東方永夜抄』(ZUN、2004)
    \item \textbf{マインドスターマイン}【近眼花火】
    \item \textbf{マインドブローイング}【赤眼催眠】『東方永夜抄』(ZUN、2004)
    \item \textbf{まえがき}【冠詞】『武蔵野話』(斎藤鶴磯、1815)
    \item \textbf{まじ}【真】『欲望我慢スル事ナカレ』
    \item \textbf{マドリガーレ}【恋歌】『黒死館殺人事件』(小栗虫太郎、1934)
    \item \textbf{マリック}【超魔術師】『小説の小説』(2022、似鳥鶏)
    \item \textbf{まんなか}【中間】『武蔵野話』(斎藤鶴磯、1815)
\section*{み}
    \item \textbf{みぎ}【茲】『武蔵野話』(斎藤鶴磯、1815)
    \item \textbf{みずち}【川祇】(2023、\UTF{56FE}\UTF{5E08})
    \item \textbf{みぞろけ}【御菩薩池】◇苗字
    \item \textbf{みたらい}【御手洗】◇苗字
    \item \textbf{みち}【街道】『武蔵野話』(斎藤鶴磯、1815)
    \item \textbf{ミッシェル・ガン・エレファント}【将棋】『ハチワンダイバー』
    \item \textbf{みなもと}【源泉】『武蔵野話』(斎藤鶴磯、1815)
    \item \textbf{ミラクルキャノン}【二重大祓砲】『呪術廻戦』(芥見下々)
\section*{む}
    \item \textbf{むかし}【昔時】『武蔵野話』(斎藤鶴磯、1815)
\section*{め}
    \item \textbf{メイキャップ}【武装】『ギラギラ』(てにをは、2021)
    \item \textbf{メエルヘン}【お伽噺】『黒死館殺人事件』(小栗虫太郎、1934)
    \item \textbf{メス}【刀】『黒死館殺人事件』(小栗虫太郎、1934)
    \item \textbf{メ・セグ}【解呪】『小説の小説』(2022、似鳥鶏)
    \item \textbf{メンタルアウト}【心理掌握】(鎌池和馬)
\section*{も}
    \item \textbf{もうまんたい}【無問題】『修道少女と偶像少女』(cosMo(暴走P)、2012)
    \item \textbf{もじ}【文】『武蔵野話』(斎藤鶴磯、1815)
    \item \textbf{もと}【基礎】「国勢調査ビラ」(東京市役所、1920?)
    \item \textbf{もの}【病気】『(バルザック)』(中原中也)
    \item \textbf{ものがたり}【音楽】『Sadistic.Music∞Factory』(cosMo(暴走P)、2013)
    \item \textbf{もみじ}【紅葉】◇
    \item \textbf{モンド}【尺】『小説の小説』(似鳥鶏、2022)
\section*{や}
    \item \textbf{やうす}【趣】『武蔵野話』(斎藤鶴磯、1815)
    \item \textbf{やしろ}【小祠】『武蔵野話』(斎藤鶴磯、1815)
    \item \textbf{やしろ}【叢祠】『武蔵野話』(斎藤鶴磯、1815)
    \item \textbf{やつ}【標的】\ccite{紅蓮の弓矢}{≪\uline{標的}≫が息絶えるまで何度でも放つ(Revo、2014)}
    \item \textbf{やまとだけのみこと}【武尊】『武蔵野話』(斎藤鶴磯、1815)
    \item \textbf{やまぶし}【修験】『武蔵野話』(斎藤鶴磯、1815)
    \item \textbf{やまぶし}【修験者】『武蔵野話』(斎藤鶴磯、1815)
    \item \textbf{ユアワース}【命を響く石】『メイドインアビス』(つくしあきひと)
    \item \textbf{ゆめ}【悪夢】『神々の祈り』(憐歌、2009)
    \item \textbf{よく}【物欲】『修道少女と偶像少女』(cosMo(暴走P)、2012)
    \item \textbf{よつぴて}【徹夜】『四方山の話 : 随筆』(三浦藤作、1927)
    \item \textbf{よのうつりかわるさま}【世推徙】(\UTF{56FE}\UTF{5E08}、2023)
    \item \textbf{ラーテル}【大驢馬】『小説の小説』(似鳥鶏、2022)
    \item \textbf{ラストダイブ}【絶界行】『メイドインアビス』(つくしあきひと)
    \item \textbf{ラブレター}【恋歌】『から紅の恋歌』(2017)
    \item \textbf{リスク}【危険性】『紅蓮の弓矢』(Revo、2014)
    \item \textbf{リンピイ}【跛者】『踊る地平線:08 しっぷ・あほうい!』(谷譲次、1934)
    \item \textbf{ルナティックレッドアイズ}【幻朧月睨】
    \item \textbf{ルナメガロポリス}【栄華之夢】『東方花映塚』
    \item \textbf{レギュレーション}【規則】『リアル初音ミクの消失』(cosMo(暴走P)、2018)
    \item \textbf{レクイエム}【鎮魂歌】『探偵たちの鎮魂歌』(2006)
    \item \textbf{レッドヘリング}【目くらまし】『小説の小説』(似鳥鶏、2022)
    \item \textbf{レンズ}【眼】『R.I.Pゴシップの海』(cosMo(暴走P)、2020)
    \item \textbf{ローゼン}【薔薇】『黒死館殺人事件』(小栗虫太郎、1934)
    \item \textbf{ローダーシュレールレ}【少女向け物語】『小説の小説』(似鳥鶏、2022)
    \item \textbf{ロスト・シップ}【難破船】『天空の難破船』(2010)
    \item \textbf{ワイバーン}【二足翼竜】『小説の小説』(2022、似鳥鶏)
    \item \textbf{ワイルド・イマジネイション}【荒唐無稽趣味】『踊る地平線:08 しっぷ・あほうい!』(谷譲次、1934)
    \item \textbf{ワイン}【葡萄酒】◇
    \item \textbf{わずかにぱらついたゆき}【片雪】
    \item \textbf{わずらい}【病気】『私本太平記:12 湊川帖』(吉川英治、1990)
    \item \textbf{わたし}
    \begin{enumerate}[label= \raise0.2ex\hbox{\textcircled{\scriptsize{\arabic*}}}]
        \item 【猫】『貴方は猫の下僕です ~ねことげぼくのヒミツなカンケイ~』(大田優一)
        \item 【一般人】\ccite{らき☆すた1}{やっぱり筋金入りのする事は\uline{一般人}には分からーん!!(美水かがみ、2005)}
    \end{enumerate}
    \item \textbf{ゑた}【帳里】『武蔵野話』(斎藤鶴磯、1815)
    \item \textbf{をなご}【婦女子】『武蔵野話』(斎藤鶴磯、1815)
\end{enumerate}

\end{document}